\documentclass[a4paper,10pt]{article}
\usepackage{latexsym}
\usepackage[polish]{babel}
\usepackage[utf8]{inputenc} 
\usepackage[MeX]{polski}
\usepackage{float}
\usepackage{geometry}
\usepackage[table]{xcolor}
\usepackage{blindtext,titlefoot}
\usepackage{wrapfig}
\usepackage{multirow}
\usepackage{array}
\usepackage{longtable}

\geometry{
	a4paper,
	total={170mm, 257mm},
	left=10mm,
	top=15mm,
}

\date{}

\begin{document}
	
	\noindent
	\textbf{\LARGE{Zestawienie wg jednostki kierującej}}\\
	
	\noindent
		Okres: od 2018-01-01 do 2018-01-31\\	
		Jednostka wykonująca: Pracownia USG\\
	
	
		\begin{longtable}[l]{| l | l | l | l|}	
			\hline
			\textbf{L.p.} & \textbf{Jednostka kierująca} & \textbf{Ilość badań} & \textbf{Wartość} \\ \hline
			\endhead
			1 & Oddział chirurgiczny & 12 & 50.00 zł \\ \hline
			\multicolumn{2}{|r|}{\textbf{Suma}} & \textbf{122} & \textbf{1450 zł} \\ \hline
		\end{longtable}
	\newpage
	
	\noindent
	\textbf{\LARGE{Zestawienie wg usługi}}\\
	
	\noindent
	Okres: od 2018-01-01 do 2018-01-31\\	
	Jednostka wykonująca: Pracownia USG\\
	
	
	\begin{longtable}[l]{| l | l | l | l| l |}	
		\hline
		\textbf{L.p.} & \textbf{Kod usługi} & \textbf{Nazwa usługi} & \textbf{Wartość} & \textbf{Ilość} \\ \hline
		\endhead
		1 & USG-4 & USG jamy brzusznej & 12 & 50.00 zł \\ \hline
		\multicolumn{3}{|r|}{\textbf{Suma}} & \textbf{122} & \textbf{1450 zł} \\ \hline
	\end{longtable}
	\newpage
	
	\noindent
	\textbf{\LARGE{Zestawienie wg jednostki kierującej z podziałem na usługi}}\\
	
	\noindent
	Okres: od 2018-01-01 do 2018-01-31\\	
	Jednostka wykonująca: Pracownia USG\\
	
	
	\begin{longtable}[l]{| l | l | l | l| l |}	
		\hline
		\textbf{L.p.} & \textbf{Jednostka kierująca} & \textbf{Usługa} & \textbf{Ilość badań} & \textbf{Wartość} \\ \hline
		\endhead
		1 & Oddział chirurgiczny & USG-4 & 12 & 50.00 zł \\ \hline
		\multicolumn{3}{|r|}{\textbf{Suma}} & \textbf{122} & \textbf{1450 zł} \\ \hline
	\end{longtable}
	
	\unmarkedfntext{Wygenerowano za pomocą programu Computerman Raporty RTG.}
\end{document}